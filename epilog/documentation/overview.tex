%%%%%%%%%%%%%%%%%%%%%%%%%%%%%%%%%%%%%%%%%%%%%%%%%%%%%%%%%%%%%%%%%%%%%%%%%%%%%%%%
\font\eightrm=cmr10 scaled 800
\font\eightbf=cmb10 scaled 800
\font\twelverm=cmr10 scaled \magstep1
\font\twelvebf=cmbx10 scaled \magstep1
\font\fourteenrm=cmr10 scaled \magstep2
\font\fourteenbf=cmbx10 scaled \magstep2
\def\chapter#1#2{\bigskip
\hfil{\twelvebf Chapter #1}
\medskip
\hfil{\fourteenbf #2}
\bigskip
\def\thechapter{#1}
\sectioncount=0}
\def\appendix#1#2{\bigskip
\hfil{\twelvebf Appendix #1}
\medskip
\hfil{\fourteenbf #2}
\bigskip
\def\thechapter{#1
\sectioncount=0}}
\def\nochapter#1{\centerline{\fourteenbf #1}\bigskip\sectioncount=0}
\def\thechapter{1}
\countdef\sectioncount=11
\countdef\subsectioncount=13
\countdef\subsubsectioncount=15
\sectioncount=0
\subsectioncount=0
\subsubsectioncount=0
\def\section#1{\advance\sectioncount by 1\subsectioncount=1
\bigskip\noindent{\bf\S\thechapter.\the\sectioncount\ #1}\par
\nobreak\medskip}
\def\subsection#1{\advance\subsectioncount by 1
\bigskip\noindent{\bf#1}\par\nobreak\medskip}
\def\nosection#1{\bigskip\noindent{\bf#1}\par\nobreak\medskip}
\def\sect#1{\advance\sectioncount by1\subsectioncount=0
\bigskip\noindent{\bf\the\sectioncount. #1}\par\medskip}
\def\subsect#1{\advance\subsectioncount by 1\subsubsectioncount=0
\bigskip\noindent{\bf\the\sectioncount.\the\subsectioncount\ #1}\par
\nobreak\medskip}
\def\subsubsect#1{\advance\subsubsectioncount by 1
\bigskip\noindent
{\bf\the\sectioncount.\the\subsectioncount.\the\subsubsectioncount\ #1}\par
\nobreak\medskip}
\def\nosect#1{\bigskip\noindent{\bf#1}\par\nobreak\medskip}
\def\heading#1{\bigskip\noindent{\bf#1}\par\medskip}
\countdef\equationcount=17
\equationcount=0
\def\equation{\global\advance\equationcount by 1
\thechapter.\the\equationcount}
\def\eq{\global\advance\equationcount by 1
\the\equationcount}
\countdef\count=19
\count=0
\def\theorem#1#2{\bigskip\noindent{\bf#1:\ }{\it#2}\par\medskip}
\def\proof{\medskip\noindent{\bf Proof:\ }}
\def\qed{\vtop{\hrule height 10pt width 5pt\bigskip}}
\def\uncatcodespecials{\def\do##1{\catcode`##1=12}\dospecials}
\def\setupverbatim{\tt\def\par{\leavevmode\endgraf}\catcode`\`=\active
\obeylines\uncatcodespecials\obeyspaces}
{\catcode`\`=\active \gdef`{\relax\lq}}
{\obeyspaces\global\let =\ }{\obeylines\global\let^^M=\par}
\def\beginverbatim{\par\begingroup\parindent=0pt\setupverbatim\doverbatim}
{\catcode`|=0 \catcode`\\=12
 |obeylines|gdef|doverbatim^^M#1\endverbatim{#1|endgroup}}
\def\verbatim{\begingroup\setupverbatim\doverb}
\def\doverb#1{\def\next##1#1{##1\endgroup}\next}
\def\start{\ }
\def\bibitem#1#2{\medskip\noindent}
\def\cite#1{[#1]}
\def\date{\the\day\ \ifcase\month\or January\or February\or March\or
April \or May\or June\or July\or August
\or September\or October\or November\or December\fi\ \the\year}
%%%%%%%%%%%%%%%%%%%%%%%%%%%%%%%%%%%%%%%%%%%%%%%%%%%%%%%%%%%%%%%%%%%%%%%%%%%%%%%%
\magnification=\magstep1
\font\bigrm=cmr10 scaled \magstep1
\def\epilog{E{\eightrm PILOG}}
\def\prolog{P{\eightrm ROLOG}}
\def\lisp{C{\eightrm OMMON} L{\eightrm ISP}}
\def\up{$\uparrow$}
%%%%%%%%%%%%%%%%%%%%%%%%%%%%%%%%%%%%%%%%%%%%%%%%%%%%%%%%%%%%%%%%%%%%%%%%%%%%%%%%

\centerline{\bf An Overview of EPILOG 2.0 for LISP}
\centerline{\bf Epistemics Inc.}

\bigskip

\sect{Introduction}

\epilog{} is a library of Common Lisp subroutines for use in programs that
manipulate information encoded in Standard Information Format (SIF), a variant
of first order predicate calculus.  It includes translators to convert
expressions from one form to another, pattern matchers of various sorts,
subroutines to create and maintain SIF knowledge bases, and a sound and complete
inference procedure based on model elimination.

The inference procedure used in \epilog{} is based on a technique called {\it
model elimination}.  The procedure closely resembles that of \prolog{}; but,
unlike that that of \prolog{}, the procedure used in \epilog{} is sound and
complete for the entire language, i.e. all consequences the system derives are
correct and it can derive all correct consequences of the information it is
given.

SIF, the language supported by \epilog{}, is a proper subset of KIF (Knowledge
Interchange Format), i.e. all expressions in SIF are expressions in KIF, but not
all KIF expressions are expressions in SIF.  Despite this subset relationship,
SIF is fully expressive, i.e. for any set of KIF sentences, there is an
equivalent set of SIF sentences.  Thus, using the subroutines in \epilog{}, it is
possible to build a sound and complete inference procedure for all of KIF.

This document is a brief introduction to some of the key features of \epilog{}. 
For more detailed information, the reader is referred to the \epilog{} reference
manual.

\sect{Standard Information Format}

SIF is a prefix version of the language of first order predicate calculus with
various extensions to enhance its expressiveness.  In \epilog{}, SIF expressions
are represented as \lisp{} atoms and lists (but not dotted pairs).

The basic vocabulary of the language includes variables, logical operators,
and constants.  Individual variables are distinguished by the presence of {\tt
?} as initial character, and sequence variables are distinguished by the
presence of an initial {\tt @}.  There is a fixed set of operators ({\tt =},
{\tt /=}, {\tt listof}, {\tt quote}, {\tt not}, {\tt and}, {\tt or}, {\tt =>},
{\tt <=}).  All others words are constants.

From this basic vocabulary, we can build {\it terms} to refer to objects in the
universe of discourse, as in the following examples.

\medskip
\beginverbatim
    (father art)
    (+ 2 3)
\endverbatim
\medskip

The language is distinctive in its use of a quotation operator to write terms
that refer to expressions.  For example, the following term refers to the
expression {\tt (+ 2 3)} literally.

\medskip
\beginverbatim
    (quote (+ 2 3))
\endverbatim
\medskip

From terms, we can build sentences.  First and foremost this provides the
ability to encode simple data, as in the examples shown below.

\medskip
\beginverbatim
    (parent art bob)
    (parent art bea)
    (parent bob cal)
\endverbatim
\medskip

We can use the {\tt =} and {\tt /=} operators to write equations and
inequalities.

\medskip
\beginverbatim
    (= (+ 2 2) 4)
    (= (father art) joe)
    (/= (quote (+ 2 3)) (+ 2 3))
\endverbatim
\medskip

Using the logical operators in SIF, it is possible to encode more complex sorts
of information (such as negations, disjunctions, rules, and so forth).  The
expression shown below is an example of a logical sentence in SIF.  It defines a
grandparent as a parent of a parent.

\medskip
{\tt (<= (grandparent ?x ?z) (parent ?x ?y) (parent ?y ?z))}\par
\medskip

\sect{Knowledge Base Manipulation}

\epilog{} provides capabilities for managing knowledge bases of sentences
encoded in SIF.  In this section, we introduce the subroutines for creating,
examining, and destroying knowledge bases.  We also look at \epilog{}'s
heterarchical theory mechanism.

The {\tt save} command is used to add facts to a specified ``theory''. Here, we
add the facts that Art is the parent of Bob, Bea, and Bess to the theory named
{\tt global}.

\medskip
\beginverbatim
User: (save '(parent art bob) 'global)
Lisp: (PARENT ART BOB)

User: (save '(parent art bea) 'global)
Lisp: (PARENT ART BEA)

User: (save '(parent art bess) 'global)
Lisp: (PARENT ART BESS)
\endverbatim
\medskip

Once we have some facts in a theory, we can examine its contents.  The {\tt
knownp} command takes a sentence and a theory as arguments.  If the sentence is
ground, then {\tt knownp} succeeds if and only if the theory contains that
sentence.  If the sentence includes individual or sequence variables, {\tt
knownp} succeeds if and only if the theory contains a sentence that matches the
argument for some binding of the constituent variables. 

\medskip
\beginverbatim
User: (knownp '(parent art bob) 'global)
Lisp: T

User: (knownp '(parent ?x bob) 'global)
Lisp: T

User: (knownp '(parent @l) 'global)
Lisp: T
\endverbatim
\medskip

The {\tt knownx} command takes an expression, a sentence, and a theory as
arguments.  If the theory contains a matching sentence, {\tt knownx} plugs the
variable bindings (if any) into the specified expression and returns the answer. 
There is no restriction on the position of the variables in the query.  Note that
there are three possible answers for the second query.  The {\tt knownx}
subroutine always returns the first answer it finds.  

\medskip
\beginverbatim
User: (knownx '?x '(parent ?x bob) 'global)
Lisp: ART

User: (knownx '?y '(parent art ?y) 'global)
Lisp: BOB

User: (knownx '(related @l) '(parent @l) 'global)
Lisp: (RELATED ART BOB)
\endverbatim
\medskip

The {\tt knowns} subroutine is similar to {\tt knownx} except that it returns a
list of all possible answers.  The order of answers on the list is the order in
which they are found in the theory.

\medskip
\beginverbatim
User: (knowns '?y '(parent art ?y) 'global)
Lisp: (BOB BEA BESS)

User: (knowns '(related ?x ?y) '(parent ?x ?y) 'global)
Lisp: ((RELATED ART BOB) (RELATED ART BEA) (RELATED ART BESS))
\endverbatim
\medskip

The {\tt knowng} subroutine takes the same arguments as {\tt knownx} and {\tt
knowns} and returns an answer generator as value.  Each time this generator is
called, it returns a different answer to the orginal question.  When all
answers have been exhausted, it return {\tt nil}. 

\medskip
\beginverbatim
User: (setq gen (knowng '?y '(parent art ?y) 'global))
Lisp: #<continuation23>

User: (funcall gen)
Lisp: BOB

User: (funcall gen)
Lisp: BEA

User: (funcall gen)
Lisp: BESS

User: (funcall gen)
Lisp: NIL
\endverbatim
\medskip

The {\tt drop} subroutine is used to delete sentences from a theory.  If {\tt
drop} succeeds in finding a matching sentence to remove, it returns the sentence
as value.

\medskip
\beginverbatim
User: (drop '(parent art bess) 'global)
Lisp: (PARENT ART BESS)

User: (knowns '?y '(parent art ?y) 'global)
Lisp: (BOB BEA)
\endverbatim
\medskip

By specifying different theory arguments to these subroutines, we can manipulate
theories without affecting other theories.  For example, here we add some
sentences to a new theory, called {\tt mytheory}.  Notice that only the new
information is available in this new theory.

\medskip
\beginverbatim
User: (save '(parent art bill) 'mytheory)
Lisp: (PARENT ART BILL)

User: (save '(parent art betty) 'mytheory)
Lisp: (PARENT ART BETTY)

User: (knowns '?y '(parent art ?y) 'mytheory)
Lisp: (BILL BETTY)
\endverbatim
\medskip

Another way of creating a theory is to use the {\tt deftheory} command.  This
subroutine takes a theory name and a list of sentences as arguments and arranges
that the theory contains exactly those sentences specified and no more.  Here, we
create a theory called {\tt another} with two more facts about Art.

\medskip
\beginverbatim
User: (deftheory another
User:   (parent art ben)
User:   (parent art barbara))
Lisp: ANOTHER

User: (knowns '?y '(parent art ?y) 'another)
Lisp: (BEN BARBARA)
\endverbatim
\medskip

Of course, we can still access our old theory by passing its name as argument. 
In the following example, we access our old data by specifying {\tt global} in
our call to {\tt knowns}.

\medskip
\beginverbatim
User: (knowns '?y '(parent art ?y) 'global)
Lisp: (BOB BEA)
\endverbatim
\medskip

Often in working with theories, it is useful to include the facts from one
theory inside of another theory.  We can do this by using the {\tt save} command
to add them in the other theory as well, but this is wasteful.  An alternative is
to call the {\tt includes} routine on two theories, thereby telling the database
routines that the first theory implicitly includes the second.  Here, we say
that {\tt mytheory} includes {\tt global} and {\tt another} and thereby make all
of the facts in these theories available whenever a subroutine accesses {\tt
mytheory}.  The {\tt includees} subroutine returns a list of all theories
included in the theory specified as argument.  The variable {\tt *includers*}
always contains a list of theories with included theories.

\medskip
\beginverbatim
User: (includes 'mytheory 'global)
Lisp: DONE

User: (knowns '?y '(parent art ?y) 'mytheory)
Lisp: (BILL BETTY BOB BEA)

User: (includees 'mytheory)
Lisp: (GLOBAL)

User: *includers*
Lisp: (MYTHEORY)
\endverbatim
\medskip

The variable {\tt *theories*} contains a list of theories containing one or more
sentences.  {\tt empty} removes all sentences from a given theory.  {\tt reset}
resets the state of the database and all variables in \epilog{} to
their initial values.

\medskip
\beginverbatim
User: *theories*
Lisp: (ANOTHER MYTHEORY GLOBAL)

User: (empty 'global)
Lisp: DONE

User: *theories*
Lisp: (ANOTHER MYTHEORY)

User: (reset)
Lisp: DONE
\endverbatim

\sect{Inference}

The basic reasoning program in \epilog{} is an efficient implementation
of the model elimination proof procedure.  The procedure is enhanced to handle
metalevel information, and there are also some extensions to support procedural
attachment and nonmonotonic reasoning.  In this section, we take a look at the
basic first order reasoning capabilities;  the metalevel reasoning ability is
described in the next section; and procedural attachments and nonmonotonic
reasoning are discussed in the section thereafter.

The special strength of the inference subroutines is their ability to do
inference with logical information encoded as implications.  Here, we enter a
definition for the {\tt grandparent} relation, and we enter some facts about
Art's family.  Although, according to our definitions, Art is the grandparent of
Cal, {\tt knownp} answers {\tt nil}.  This is the correct answer for {\tt knownp}
-- after all, the fact is not stored explicitly in the theory.  By contrast, {\tt
findp} is able to prove the fact.  The {\tt findx} subroutine is able to find a
grandparent of Cal and a grandchild of Art.  The {\tt finds} subroutine is
able to find all of the grandchildren of Art.  The {\tt findg} subroutine
returns a generator.

\medskip
\beginverbatim
User: (save '(<= (grandparent ?x ?z) (parent ?x ?y) (parent ?y ?z))
            'global)
Lisp: (<= (GRANDPARENT ?X ?Z) (PARENT ?X ?Y) (PARENT ?Y ?Z))

User: (save '(parent art bob) 'global)
Lisp: (PARENT ART BOB)

User: (save '(parent bob cal) 'global)
Lisp: (PARENT BOB CAL)

User: (save '(parent bob coe) 'global)
Lisp: (PARENT BOB COE)

User: (knownp '(grandparent art cal) 'global)
Lisp: NIL

User: (findp '(grandparent art cal) 'global)
Lisp: T

User: (findx '?x '(grandparent ?x cal) 'global)
Lisp: ART

User: (findx '?y '(grandparent art ?y) 'global)
Lisp: CAL

User: (finds '?y '(grandparent art ?y) 'global)
Lisp: (CAL COE)

User: (setq gen (findg '?y '(grandparent art ?y) 'global))
Lisp: #<continuation24>

User: (funcall gen)
Lisp: CAL

User: (funcall gen)
Lisp: COE

User: (funcall gen)
Lisp: NIL
\endverbatim
\medskip

The inference subroutines introduced so far illustrate backward reasoning (from
the goal to premises using backward implications).  \epilog{} is also
capable of forward reasoning (from premises to conclusions using forward
implications).

As an example, consider the following interaction.  By setting {\tt *saves*} to
{\tt (family)}, the user directs the system to save literals involving the {\tt
family} relation.  The forward implication asserts that, if a person is in a
particular family, then all of his children are in that family as well.  By
writing it as a forward implication, we are saying that the implication should be
triggered whenever we get information about a person's family.  The subroutine
{\tt assume} is used to add sentences to the database and conduct all such
forward chaining.  If we discover that Art is a Garfunkel, then we immediately
conclude that Bob and Cal and Coe are Garfunkels as well and these facts are
stored explicitly in the database.

\medskip
\beginverbatim
User: (setq *saves* '(family))
Lisp: (FAMILY)

User: (save '(=> (family ?x ?z) (parent ?x ?y) (family ?y ?z)) 'global)
Lisp: (=> (FAMILY ?X ?Z) (PARENT ?X ?Y) (FAMILY ?Y ?Z))

User: (assume '(family art garfunkel) 'global)
Lisp: DONE

User: (knowns '?x '(family ?x garfunkel) 'global)
Lisp: (ART BOB CAL COE)
\endverbatim
\medskip

So far, we have concentrated exclusively on relations.  Much, if not most, of
our conceptualization of the world naturally takes the form of functions.  For
functional information, we use a notation much closer to that of \lisp{}.  If a
list is empty, the result of appending the list onto a second list is just
the second list; otherwise, the result is obtained by adding the first element
of the list to the result of appending the rest of the list to the second list.

\medskip
\beginverbatim
User: (save '(= (append (listof) ?m) ?m) 'global)

User: (save '(<= (= (append (listof ?x @l) ?m) (listof ?x @n))
                 (= (append (listof @l) ?m) (listof @n)))
            'global)
\endverbatim
\medskip

Now it is possible to append two lists by making a call to the {\tt findx}
routine.

\medskip 
\beginverbatim
User: (findx '?z '(= (append (listof 1 2) (listof 3 4)) ?z) 'global)
Lisp: (listof 1 2 3 4)
\endverbatim
\medskip

The inference subroutines also include a subroutine, called {\tt findval},
capable of evaluating terms directly.

\medskip 
\beginverbatim
User: (findval '(append (listof 1 2) (listof 3 4)) 'global)
Lisp: (listof 1 2 3 4)
\endverbatim
\medskip

Here, we have a more interesting example of automated reasoning.  Given the
axioms of equality (reflexivity, symmetry, and transitivity) and the group
axioms (left and right identities for the {\tt *} function, right inverse, and
associativity), we ask the system to prove that the right inverse is also a left
inverse.

\medskip 
\beginverbatim
User: (deftheory equality
        (<= (= ?x ?y) (== ?x ?y))
        (= ?x ?x)
        (<= (= ?x ?y) (= ?y ?x))
        (<= (= ?x ?z) (= ?x ?y) (= ?y ?z)))
Lisp: EQUALITY

User: (deftheory group
        (<= (= (* ?x ?y) ?x) (= ?y e))
        (<= (= (* ?y ?x) ?x) (= ?y e))
        (<= (= (* ?x ?y) e) (= ?y (inv ?x)))
        (<= (= (* ?x ?v) ?w)
            (= (* ?y ?z) ?v)
            (= (* ?x ?y) ?u)
            (= (* ?u ?z) ?w)))
Lisp: GROUP

User: (includes 'group 'equality)
Lisp: DONE

User: (setq *depth* 5)
Lisp: 5

User: (findp '(= (* (inv x) x) e) 'group)
Lisp: T
\endverbatim
\medskip

Problems like this one are much more complicated than the simpler examples shown
earlier.  While \epilog{} can handle such cases, the computational
cost can be quite large.  For example, the computational cost for the previous
examples is just a few milliseconds apiece, whereas the cost in this case is
about a second, and the cost to prove more complicated theorems can be much
higher.  Fortunately, the inference algorithm used in \epilog{} is
quite simple; and, as a result, it is usually possible to predict its
computational performance on a given knowledge base and to control that
performance by writing sentences in a judicious manner.

\sect{Metaknowledge}

One of the distinctive features of SIF is that it allows us to express knowledge
about knowledge.  In order to encode such {\it metaknowledge}, we use a
conceptualization in which expressions in the language are treated as objects in
the universe of discourse and in which there are functions and relations
appropriate to these objects.  In SIF, atoms are treated as primitive objects
(i.e. having no subparts).  Complex expressions (i.e. non-atoms) are treated
as lists of subexpressions (either atoms or other complex expressions).  In
particular, every complex expression is viewed as a list of its immediate
subexpressions.

In order to assert properties of expressions in the language, we need a way of
referring to those expressions.  There are two ways of doing this in SIF.  One
way is to use the {\tt quote} operator in front of an expression.  For example,
to refer to the symbol {\tt john}, we use the term {\tt 'john} or, equivalently,
{\tt (quote john)}.  To refer to the expression {\tt (p a b)}, we use the term
{\tt '(p a b)} or, equivalently, {\tt (quote (p a b))}.

With a way of referring to expressions, we can assert their properties.  For
example, the following sentence ascribes to the individual named {\tt john} the
belief that the moon is made of a particular kind of blue cheese.

\medskip
\beginverbatim

User: (save '(believes john '(material moon stilton)) 'global)
Lisp: (BELIEVES JOHN '(MATERIAL MOON STILTON))
\endverbatim
\medskip

Since expressions are first-order objects, we can quantify over them, thereby
asserting properties of whole classes of sentences.  For example, we could say
that Mary believes everything that John believes.  This fact together with the
preceding fact allows us to conclude that Mary also believes the moon to be made
of blue cheese.

\medskip
\beginverbatim
User: (save '(<= (believes mary ?p) (believes john ?p)) 'global)
Lisp: (<= (BELIEVES MARY ?P) (BELIEVES JOHN ?P))

User: (findp '(believes mary '(material moon stilton)) 'global)
Lisp: T
\endverbatim
\medskip

The second way of referring to expressions is SIF is to use the {\tt listof}
operator.  We can denote a complex expression like {\tt (p a b)} by a term of
the form {\tt (listof 'p 'a 'b)}, as well as {\tt '(p a b)}.  All of \epilog{}'s
subroutines treat the {\tt quote} and {\tt listof} forms of denotation as
equivalent.  If variables are included in the list form, then any attempt to
match the two forms will result in the variables being bound to the appropriate
subexpressions.

\medskip
\beginverbatim
User: (findp '(believes mary (listof 'material 'moon 'stilton)) 'global)
Lisp: T

User: (findx '?y '(believes mary (listof 'material 'moon ?y)) 'global)
Lisp: 'STILTON
\endverbatim
\medskip

Unfortunately, these {\tt listof} expressions can be quite cumbersome.  In order
to reduce this complexity, SIF defines the read macro characters {\tt \up} and
{\tt ,} to assist the user in writing such expressions in a more natural form. 
The use of {\tt \up} signals thats the following expression is to be quoted {\it
except} for those components preceded by commas.

\medskip
\beginverbatim
User: (findx '?y '(believes mary ^(material moon ,?y)) 'global)
Lisp: 'STILTON
\endverbatim
\medskip

The advantage of the {\tt listof} representation over the {\tt quote}
representation is that it allows us to quantify over parts of expressions.  For
example, let us say that Lisa is more skeptical than Mary.  She agrees with
John, but only on the composition of the moon.  The first sentence below asserts
this fact without specifically mentioning {\tt stilton}.  Thus, if we were to
later discover that John thought the moon to be made of rocks, then Lisa would
be constrained to believe this as well.  The second sentence relates our
assertion about Lisa's beliefs to the world of real objects, not just symbols.

\medskip
\beginverbatim
User: (save '(<= (believes lisa ^(material moon ,?y))
                 (believes john ^(material moon ,?y))) 'global)

User: (findp '(believes lisa '(material moon stilton)) 'global)
Lisp: T
\endverbatim
\medskip

Another use of {\tt quote} and {\tt listof} is in the formalization of rules of
inference.  For example, the following sentence formalizes the ordered resolution
rule of inference for ground clauses.  If the first literals in two clauses are
complementary, then it is legal to conclude the clause consisting of the
remaining literals of each clause.  (Adding unification to this formalization to
handle non-ground clauses presents no serious problems.)

\medskip
\beginverbatim
User: (save '(<= (resolution ^(or ,?p ,@l)
                             ^(or (not ,?p) ,@m)
                             ^(or ,@n))
                 (= (append (listof @l) (listof @m)) (listof @n)))
            'global)
\endverbatim
\medskip

With this information, it is possible for the system to answer metalevel
questions about resolution, like the one shown.

\medskip
\beginverbatim
User: (findx '?r '(resolution '(or p q) '(or (not p) (not r)) ?r) 'global)
Lisp: (LISTOF 'OR 'Q '(NOT R))
\endverbatim
\medskip

While there is little practical value to defining resolution in this way, this
example illustrates the power of the representation and the \epilog{} routines. 
More practical examples of this include representation of knowledge about
the beliefs, goals, interests, and capabilities of agents and metainformation
useful in the control of reasoning.

\sect{Procedural Attachments}

The knowledge base manipulation routines in \epilog{} are unable to answer
even the simplest questions of arithmetic or equality (unless those facts are
explicitly stored).  By contrast, the inference subroutines have knowledge of
many basic concepts of SIF, e.g the {\tt >} concept.

\medskip
\beginverbatim
User: (findp '(> 3 2) 'global)
Lisp: T
\endverbatim
\medskip

Predefined terms are evaluated using the {\tt ==} relation.  In this case, the
second argument to {\tt ==} may be a variable.

\medskip 
\beginverbatim
User: (findp '(== (+ 2 2) 4) 'global)
Lisp: T

User: (findx '?x '(== (+ 2 2) ?x) 'global)
Lisp: 4
\endverbatim
\medskip

In addition to these concept-specific attachments, there is a simple but powerful
procedural attachment mechanism.  The function constant {\tt execute} takes as
argument a fragment of \lisp{}.  Whenever an equation with an {\tt execute} term
as first argument is evaluated, the specified procedure is called on the specified
arguments and the a description of the value is unified with the term occurring
as the second argument of the equation.  The following examples illustrate this
mechanism.

\medskip
\beginverbatim

User: (findp '(== (execute (listp '(a b c))) 't) 'global)
Lisp: T
\endverbatim
\medskip

Note that calls can be made to the inference routine itself and thus implement a
kind of nonmonotonic reasoning.

\medskip
\beginverbatim
User: (deftheory global
        (parent art bob)
        (parent art bea)
        (parent art bess)
        (= (length (listof)) 0)
        (<= (= (length (listof ?x @l)) ?n)
               (= (length (listof @l)) ?n1)
               (== (+ ?n1 1) ?n))
        (<= (= (numchildren ?x) ?z)
            (== (execute (finds '?y '(parent art ?y) 'global)) ?y)
            (= (length ?y) ?z)))
Lisp: DONE

User: (findval '(numchildren art) 'global)
Lisp: 3
\endverbatim
\medskip

\sect{Conclusion}

\epilog{} also includes a full range of utilities (e.g. secondary
storage handling) and debugging aids (e.g. for tracing inference routines).

Please remember that this document is just an introduction to \epilog{}. 
It is intended to be suggestive, not exhaustive.  Although the subroutines
presented here illustrate the key features of \epilog{}, there are many
other subroutines as well.  Although the examples used here are quite simple (in
keeping with the introductory nature of the presentation), \epilog{} is
capable of and intended for use in the construction of very complex
knowledge-based systems.

\bye

%%%%%%%%%%%%%%%%%%%%%%%%%%%%%%%%%%%%%%%%%%%%%%%%%%%%%%%%%%%%%%%%%%%%%%%%%%%%%%%%
%%%%%%%%%%%%%%%%%%%%%%%%%%%%%%%%%%%%%%%%%%%%%%%%%%%%%%%%%%%%%%%%%%%%%%%%%%%%%%%%
%%%%%%%%%%%%%%%%%%%%%%%%%%%%%%%%%%%%%%%%%%%%%%%%%%%%%%%%%%%%%%%%%%%%%%%%%%%%%%%%
